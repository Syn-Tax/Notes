% Created 2020-10-07 Wed 21:00
% Intended LaTeX compiler: pdflatex
\documentclass[11pt]{article}
\usepackage[utf8]{inputenc}
\usepackage[T1]{fontenc}
\usepackage{graphicx}
\usepackage{grffile}
\usepackage{longtable}
\usepackage{wrapfig}
\usepackage{rotating}
\usepackage[normalem]{ulem}
\usepackage{amsmath}
\usepackage{textcomp}
\usepackage{amssymb}
\usepackage{capt-of}
\usepackage{hyperref}
\usepackage{indentfirst}
\usepackage[a4paper]{geometry}
\linespread{2}
\author{Oscar Morris}
\date{}
\title{'The demographic dividend has the power to transform countries economically' - to what extent do you agree with this statement?}
\hypersetup{
 pdfauthor={Oscar Morris},
 pdftitle={'The demographic dividend has the power to transform countries economically' - to what extent do you agree with this statement?},
 pdfkeywords={},
 pdfsubject={},
 pdfcreator={Emacs 27.1 (Org mode 9.4)}, 
 pdflang={English}}
\begin{document}

\maketitle
A demographic dividend is generally defined as the potential economic benefit offered by changes in the age structure of the population, usually allowing for a larger working population. However, some studies have suggested that this increase in the workforce is due more to an increase in education rather than purely an increase in the working population \cite{lutz2019}. Often this increase in education does bring about a change in the age structure of a population (due to increased awareness of family planning, increased awareness of and ability to use contraceptives and a decrease in infant mortality) which, in turn, increases the proportion of the population at a working age. This, however, isn't the only contributing factor to the economic growth seen after a fall in birth and death rates, another big factor is the reduction in the dependency ratio, between those of working age and those that are either too young or too old to be able to make a valuable contribution to the workforce.

Research has shown that these age-structural transitions (ASTs), or transitions from a more 'youthful' population to an 'older' population, can affect the economic growth of a country. This economic growth often occurs because during an AST, public spending that has previously been allocated to social programmes (health, education) can now be directed towards more productive sectors and the improvement of infrastructure \cite{pool2007}. On a smaller scale, families can direct more of their spending towards increased savings and therefore, often vastly, improved living standards. This is often directly tied to families being smaller and therefore less expensive for the parents. An increase in education which often follows an AST will also increase litracy rates and general education allowing a higher proportion of the population to have more highly skilled, and therefore valued, professions.

The issue with these demographic dividends is that the policy makers, usually a government, must create policies to harness the potential of such a dividend. This is the most critical part of the dividend period as if the current employment infrastructure cannot handle the large influx of people above working age then the unemployment rate will rise and so will the dependency ratio therefore reducing and possibly halting economic growth completely. This is the big challenge now facing sub-Saharan and central Africa because these regions are estimated to account for almost 80 percent of the projected 4 billion increase in global population. This increase in the working-age population must be harnessed by each country in these regions independently to achive the highest possible economic growth \cite{drummond2014}.

One of the examples of a country that is experiencing an AST and a dividend is Bangladesh where there is currently a large increase in the working-age population and a prominent decrease in the dependent-age population, decreasing the dependency ratio and allowing for economic growth using the factors outlined above. This dividend period started in 1980 and is predicted to continue for a period of 60 years, until 2040. It is currently important for Bangladesh to provide policies regarding child labour and education for children, along with public health and those that promote labour market flexibility and provide incentives for investment and saving. On the other hand, if appropriate policies are not introduced, this dividend could, become a cost leading to unemployment and high amounts of strain on education and health \cite{matin2012}.

In conclusion, a demographic dividend does have the potential to economically transform countries. However, a dividend can only be harnessed and transferred into economic growth if the correct legislation is provided. If appropriate legislation is not provided then this dividend can become a cost to the country's economy with high amounts of unemploymnet and strain on public services.

\bibliographystyle{apalike}
\bibliography{demographic_dividend}
\end{document}

% Created 2020-09-17 Thu 21:35
% Intended LaTeX compiler: pdflatex
\documentclass[11pt]{article}
\usepackage[utf8]{inputenc}
\usepackage[T1]{fontenc}
\usepackage{graphicx}
\usepackage{grffile}
\usepackage{longtable}
\usepackage{wrapfig}
\usepackage{rotating}
\usepackage[normalem]{ulem}
\usepackage{amsmath}
\usepackage{textcomp}
\usepackage{amssymb}
\usepackage{capt-of}
\usepackage{hyperref}
\usepackage[backend=bibtex,style=verbose-trad2]{biblatex}
\author{<Your Name Here}
\date{\today}
\title{Test}
\hypersetup{
 pdfauthor={<Your Name Here},
 pdftitle={Test},
 pdfkeywords={},
 pdfsubject={},
 pdfcreator={Emacs 27.1 (Org mode 9.4)}, 
 pdflang={English}}
\begin{document}

\maketitle
\tableofcontents



\section{Abstract}
\label{sec:org65ab663}

You cannot use an org-mode header here.
If you do, it trashes the table of contents for the apa6 document class.
That's why Abstract is bolded manually.

As you can see, I write my documents 1 sentence to a line.
This is because I keep these documents under version control.
A single English sentence is similar to a single line of code.
You wouldn't run lines of code together in a production codebase, so don't run sentences together in a VC'ed text document.

Latex and org-mode both interpret a single empty line as a paragraph break, so the fact that your source document is 1 sentence per line will not be visible to anybody other than you.

\cite{devlin2019}
\section{Bibliography}
\label{sec:org531805b}
\bibliographystyle{ieeetr}
\bibliography{bibliography}
\printbibliography
\end{document}

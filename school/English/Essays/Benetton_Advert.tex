% Created 2020-10-06 Tue 20:48
% Intended LaTeX compiler: pdflatex
\documentclass[11pt]{article}
\usepackage[utf8]{inputenc}
\usepackage[T1]{fontenc}
\usepackage{graphicx}
\usepackage{grffile}
\usepackage{longtable}
\usepackage{wrapfig}
\usepackage{rotating}
\usepackage[normalem]{ulem}
\usepackage{amsmath}
\usepackage{textcomp}
\usepackage{amssymb}
\usepackage{capt-of}
\usepackage{hyperref}
\usepackage{indentfirst}
\author{Oscar Morris}
\date{}
\title{Benetton Adverts}
\hypersetup{
 pdfauthor={Oscar Morris},
 pdftitle={Benetton Adverts},
 pdfkeywords={},
 pdfsubject={},
 pdfcreator={Emacs 27.1 (Org mode 9.4)}, 
 pdflang={English}}
\begin{document}

\maketitle

\section{Newborn child}
\label{sec:org0c6ea81}
The Benetton advert showing a new born child is likely one of it's most controvertial adverts and almost certainly the company's most censored. It depicts a newborn baby, ``Giusy'', still attached to the umbilical cord. According to Benetton it was intended to represent an ``anthem to life'' however it wasn't recieved so well with customers and the local councils, some of which decided to force Benetton to remove it.

Regarding the image as a whole it is making good use of the rule of thirds with both the baby's feet and the ``UNITED COLORS OF BENETTON'' logo. This would lead the eye to either the baby's feet or it's head and then along it's body. The colour used for the person holding the baby as moro of a blue-ish gray makes it less of an object of focus and moving the eye to the baby. The colour used in the Benetton logo is also fairly vibrant with a high amount of contrast between the dark green in the background of the logo and the white text allowing it to be read from further away, the font, being a sans-serif font and therefore easier to read only adds to this.

\section{World leaders kissing}
\label{sec:orgf29cb3a}
Another of Benetton's more controvertial adverts depicts the former President Barack Obama kissing his Chinese conuterpart Xi Jinping on the left and Pope Benedict XVI kissing the prominent Egyptian sunni muslim, Ahmed el Tayeb. These adverts were part of Benetton's ``unhate'' campaign. This advertising campaign was intended to promote tolerance to many forms of discrimitation.

The contrast between colours is again prominent with the black and white of the Pope's and Tayeb's clothing, and the similarities in Obama's and Jinping's clothing indicating a similar profession (both being political leaders) or implying that clothes bring people together. The word ``unhate'' is interesting as each syllable is written in a different font and style, the 'un' is written in a normal weight, serif font whereas the 'hate' is written in bold, italics and a sans-serif font, this brings emphasis on the 'hate' and could indicate aggression whereas the serif font of the 'un' could indicate a sort of importance and elegance to it. The 'un' could also represent the United Nations which represent every nation, race and religion and should therefore be without discrimination. This could also reflect that Benetton's clothes are supposed to be worn by anyone regardless of sexuality, race or religion.
\section{Conclusion}
\label{sec:org6361c4b}
In conclusion both adverts are among the most controvertial and censored of all time and ultimately did increase the knowledge of the brand however it was mostly for the reason that these adverts were controvertial. Benetton are however spreading some important messages with their 'unhate' campaign such as fighting discrimination of races, sexuality and religion.
\end{document}
